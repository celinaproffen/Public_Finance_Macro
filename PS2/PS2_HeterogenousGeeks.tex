\documentclass[12pt,a4paper]{article}

\usepackage[a4paper, top = 2cm, bottom = 2cm, left = 1.5cm, right = 1.5cm]{geometry}
\usepackage[dvipsnames]{xcolor} % colors
\usepackage{amsthm}
\usepackage{xcolor}
\usepackage{standalone}
\usepackage{physics}
\usepackage{setspace}
\usepackage{graphicx}
\usepackage{amsfonts}
\usepackage{amsmath}
\usepackage{tikz}
\usepackage{pdfpages}
\usepackage{epigraph}
\usepackage{csquotes}
\usepackage{natbib}
% Bibliography
\usepackage{xcolor}
\usepackage{hyperref}
\hypersetup{
    colorlinks=true,
    citecolor=MidnightBlue,
		linkcolor=MidnightBlue,
    pdfpagemode=FullScreen,
    }

\usepackage{natbib}
\usepackage[noabbrev]{cleveref}
\setcitestyle{authoryear,open={(},close={)}}
\bibliographystyle{plainnat}
\usepackage{subfiles}
\setlength\parindent{0pt}
\spacing{1.2}

\newcommand{\R}{\mathbb{R}}
\newcommand{\Z}{\mathbb{Z}}
\newcommand{\N}{\mathbb{N}}
\newcommand{\Q}{\mathbb{Q}}
\begin{document}
\textbf{Problem 2}:\\
1.From our regression on the cubic in age from project 4, the coefficients are estimated as:
\begin{equation*}
    ln\_yhh5\_hat=8.36501+0.0861011\cdot age-0.0006384 \cdot age^2-0.0000035\cdot age^3+0.4370217\cdot married   
\end{equation*}
\begin{equation*}
    yhh5\_hat=exp(8.36501+0.0861011\cdot age-0.0006384 \cdot age^2-0.0000035\cdot age^3+0.4370217\cdot married)  
\end{equation*}
With this we can estimate the age income profile for a given age, for simplification, we choose $married=0$. Since we're only interested in the income from age 20 to 64, we keep the estimates for this age range and set income for all other ages equal 0. After that we can normalize the age profile by dividing income of every age by the average income of life cycle. Finally, we replace the normalized age profile in the code with the normalized estimates. Please refer to $\texttt{towards\_olg\_prob\_2.m}$ for the estimates of age profile. \\

Interpretation for $j_r=45$ and $j_r=60$:

\\

2.
\begin{equation}\tag{1a}
     \Tilde{y_t}&=z_t+\epsilon_t 
\end{equation}
\begin{equation}\tag{1b}
    z_t&=\rho z_{t-1} +\nu_t
\end{equation}
Note that (1a) implies:
\begin{equation*}
    z_t=\Tilde{y_t}-\epsilon_t \hspace{5mm} \forall t
\end{equation*}
Therefore from (1b), we have:
\begin{equation*}
    z_{t}=\rho(\Tilde{y}_{t-1}-\epsilon_{t-1})+\nu_t
\end{equation*}
Plugging in (1a) and notice that $\epsilon_t$ and $\nu_t$ are i.i.d:
\begin{equation}\label{one}
    \Tilde{y_t}=\rho(\Tilde{y}_{t-1}-\epsilon_{t-1})+\nu_t+\epsilon_t
\end{equation}
Since the all shocks are orthogonal to each other and the process is weakly stationary, taking the variance yields:
\begin{align*}
    \sigma^2_{\Tilde{y}}&=\rho^2\sigma^2_{\Tilde{y}}+\sigma^2_{\nu}+(1-\rho^2)\sigma^2_{\epsilon}\\
    &=\frac{\sigma^2_\nu}{1-\rho^2} +\sigma^2_\epsilon\\
    &=\sigma^2_z+\sigma^2_\epsilon
\end{align*}
Dividing both sides by $\sigma^2_z$:
\begin{equation*}
    \frac{\sigma^2_{\Tilde{y}}}{\sigma^2_z}=1+\frac{\sigma^2_\epsilon}{\sigma^2_z}
\end{equation*}
Notice that, by i.i.d assumption of shocks from both processes: 
\begin{align*}
    cov(\Tilde{y_t},\Tilde{y}_{t-1})&=cov(\rho z_{t-1} +\nu_t+\epsilon_t,z_{t-1}+ \epsilon_{t-1})=cov(\rho z_{t-1},z_{t-1})=\rho Var(z_{t-1})=\rho\frac{\sigma^2_{\nu}}{1-\rho^2}\\
    cov(z_t,z_{t-1})&=cov(\rho z_{t-1}+\nu_t,z_{t-1})=\rho Var(z_{t-1})=\rho\frac{\sigma^2_{\nu}}{1-\rho^2}
\end{align*}
Therefore:
\LongRightarrow $cov(\Tilde{y_t},\Tilde{y}_{t-1})=cov(z_{t},z_{t-1})$(*)\\
Furthermore, we have:
\begin{align*}
    \rho_{\Tilde{y}}&=\frac{cov(\Tilde{y_t},\Tilde{y}_{t-1})}{ \sigma^2_{\Tilde{y}}}\\
    \rho&=\frac{cov(z_t,z_{t-1})}{ \sigma^2_z}\\
\end{align*}
Taking the ratio, and from (*), we have:
\begin{equation*}
  \frac{\rho_{\Tilde{y}}}{\rho}=\frac{\sigma^2_z}{\sigma^2_{\Tilde{y}}} =\bigg(1+\frac{\sigma^2_\epsilon}{\sigma^2_z}\bigg)^{-1}
\end{equation*}
Thus: 
\begin{equation*}
\rho_{\Tilde{y}}=\rho\bigg(1+\frac{\sigma^2_\epsilon}{\sigma^2_z}\bigg)^{-1}(Q.E.D)
\end{equation*} 
\textbf{Intuition}: The larger the magnitude of transitory shock relevant to persistent shock, the less variation of $\Tilde{y}$ can be explained by the variation of persistent shock as implied by \eqref{one}, therefore if there were no transitory shock, autocorrelation of $\Tilde{y}$ is equal to that of z, which is $\rho$. Otherwise, $\rho$ should be scaled down by a factor inversely proportional to the ratio of transitory shock and persistent shock, which is $\frac{1}{1+\frac{\sigma^2_\epsilon}{\sigma^2_z}}$\\
\begin{itemize}
    \item \textbf{Calibration with estimates of gross income}\\

      From estimates in project 4, $\epsilon$ can be calculated as:
\begin{equation*}
    \epsilon=\sigma_{\Tilde{y}}=\sigma_{ln \Tilde{y}}=\sqrt{\frac{\sigma^2_\nu}{1-\rho^2} +\sigma^2_\epsilon}\approx \sqrt{\frac{0.0861262}{1-0.9088825^2} +0.2144817}\approx 0.8424
\end{equation*}
The state vector $[-\epsilon,\epsilon]$ of the logs therefore is: [-0.8424,0.8424]\\
Using the unnumbered equations for $\eta_-$, $\eta_+$ on p.42 of the lecture notes, the the logs can be transformed to levels as: 
\begin{equation*}
    \eta_-=\frac{2exp(1-\sigma_{ln\Tilde{y}})}{exp(1-\sigma_{ln\Tilde{y}})+exp(1+\sigma_{ln\Tilde{y}})}\approx\frac{2exp(1-0.8424)}{exp(1-0.8424)+exp(1+0.8424)}\approx0.3129
    \end{equation*}
    \begin{equation*}
       \eta_+=\frac{2exp(1+\sigma_{ln\Tilde{y}})}{exp(1-\sigma_{ln\Tilde{y}})+exp(1+\sigma_{ln\Tilde{y}})}\approx\frac{2exp(1+0.8424)}{exp(1-0.8424)+exp(1+0.8424)}\approx 1.687 
    \end{equation*}
From previous calibration: $\rho_{\Tilde{y}}=\rho\bigg(1+\frac{\sigma^2_\epsilon}{\sigma^2_z}\bigg)^{-1}\approx0.9088825\bigg(1+\frac{0.2144817}{16.15252}\bigg)^{-1}\approx 0.897 $\\
Using $\kappa=\frac{1+\rho_{\Tilde{y}}}{2}\approx 0.9485$, transition matrix of the Markov process can be calibrated as:\\
\begin{center}
$\Pi$ = \begin{bmatrix}
0.9485 & 0.0515 \\
0.0515 & 0.9485  
\end{bmatrix}
\end{center}\\

Using the estimated values, we can input into the mchain function to implement the model. Please refer to $\texttt{towards\_olg\_prob\_2.m}$ for the implementation. 
    
    
\end{itemize}


    


    



\end{document}


