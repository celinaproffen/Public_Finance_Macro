\documentclass[12pt,a4paper]{article}

\usepackage[a4paper, top = 2cm, bottom = 2cm, left = 1.5cm, right = 1.5cm]{geometry}
\usepackage[dvipsnames]{xcolor} % Colors

\usepackage{standalone}

\usepackage{setspace}
\usepackage{graphicx}
\usepackage{amsfonts}
\usepackage{amsmath}
\usepackage{tikz}
\usepackage{pdfpages}
\usepackage{epigraph}
\usepackage{csquotes}

% Bibliography
\usepackage{xcolor}
\usepackage{hyperref}
\hypersetup{
    colorlinks=true,
    citecolor=MidnightBlue,
		linkcolor=MidnightBlue,
    pdfpagemode=FullScreen,
    }

\usepackage{natbib}
\usepackage[noabbrev]{cleveref}
\setcitestyle{authoryear,open={(},close={)}}
\bibliographystyle{plainnat}

\usepackage{subfiles}

\setlength\parindent{0pt}
\spacing{1.2}

\begin{document}

\begin{center}
       \vspace*{1cm}
       \huge\textbf{Problemset 2} \\
       \vspace{0.4cm}
       \large \textbf{Public Finance in Macroeconomics} \\
       \vspace{0.5cm}
        \large Handed in by the \textcolor{orange}{\textbf{Heterogeneous Geeks}} \\ 
        \vspace{0.3cm}
        a.k.a. Vivien Voigt, Thong Nguyen, \includegraphics[scale=0.06]{geek.png}\\Davide Difino \& Celina Proffen \\
       \vspace{1.5cm}
       \vfill
       
       % I also thought of adding some cool superheroe image her, but I didn't find one - feel free to delete the gecko if you don't like it/ find it appropriate! ;)
       
        Problem set in the context of Prof. Ludwig's course: \\
        \textbf{Public Finance in Macroeconomics: Heterogenous Agent Models}\\
        at the Graduate School of Economics Finance and Management
       \vspace{0.8cm}
   \end{center}

\newpage

%%%%%%%%%%%%%%%%%%%%%%%%%%%%%%%%%%%%%%%%%%%%%%%%%%%%%%%%%%%%%%%%%%%%
%%%%%%%%%%%%%%%%%%%%%%%%%%%%%%%%%%%%%%%%%%%%%%%%%%%%%%%%%%%%%%%%%%%%

\section*{Problem 1}

Regarding consumption note:
\begin{itemize}
    \item \textbf{food:} 
    \item \textbf{ndcons1:} 
    \item \textbf{ndcons2:} 
    \item \textbf{ndconsserv:} 
    \item \textbf{totcons:} 
\end{itemize}

%%%%%%%%%%%%%%%%%%%%%%%%%%%%%%%%%%%%%%%%%%%%%%%%%%%%%%%%%%%%%%%%%%%%
%%%%%%%%%%%%%%%%%%%%%%%%%%%%%%%%%%%%%%%%%%%%%%%%%%%%%%%%%%%%%%%%%%%%

\section*{Problem 2}

\textbf{General explanation on how we deal with the data}

\textcolor{red}{Interviews take place in a quarter BUT consumption always refers to consumption in the 3 previous months, while income refers to the income in the last 12 months!} \\

\textbf{Specifications in Mace's paper (??) and our extensions}
There are basically two specifications that we want to test (both by Mace). Each of this will receive a subsection, in which we first re-estimate her results (as well as we possibly can, as we don't observe employment status). Then we usually expand on this specification by adding additional controls or extending the model otherwise. 

%%%%%%%%%%%%%%%%%%%%%%%%%%%%%%%%%%%%%%%%%%%%%%%%%%%%%%%%%%%%%%%%%%%%
\subsection*{Hypothesis 1: Individual Consumption depends only on Aggregate Spending}
What does the hypothesis expect? 

\textbf{2.1.A. Mace's specification}
In all coming specification we will always look at the same type of consumption (i.e. food, non-durables including education, non-durables excluding education, etc.) on the individual and aggregate levels.

Table \ref{tab:2.1A-deltacons} 
\begin{table}[!h]\centering
\def\sym#1{\ifmmode^{#1}\else\(^{#1}\)\fi}
\caption{\label{tab:2.1A-deltacons} Explaining changes in consumption}
\begin{tabular}{l*{5}{c}}
\hline\hline
            &\multicolumn{1}{c}{(1)}         &\multicolumn{1}{c}{(2)}         &\multicolumn{1}{c}{(3)}         &\multicolumn{1}{c}{(4)}         &\multicolumn{1}{c}{(5)}         \\
\hline
avg-totcons-change&       1.000\sym{***}&                     &                     &                     &                     \\
            &    (0.0628)         &                     &                     &                     &                     \\
diff\_netinc &      0.0215\sym{***}&     0.00116\sym{***}&     0.00463\sym{***}&     0.00345\sym{***}&     0.00474\sym{***}\\
            &   (0.00137)         &  (0.000195)         &  (0.000498)         &  (0.000339)         &  (0.000526)         \\
avg-food-change&                     &       1.005\sym{***}&                     &                     &                     \\
            &                     &    (0.0570)         &                     &                     &                     \\
avg-ndcons1-change&                     &                     &       0.998\sym{***}&                     &                     \\
            &                     &                     &    (0.0534)         &                     &                     \\
avg-ndcons2-change&                     &                     &                     &       1.000\sym{***}&                     \\
            &                     &                     &                     &    (0.0639)         &                     \\
avg-ndconsserv-change&                     &                     &                     &                     &       0.999\sym{***}\\
            &                     &                     &                     &                     &    (0.0568)         \\
constant    &      -12.35         &      -0.645         &      -2.549         &      -1.969         &      -2.701         \\
            &     (24.92)         &     (2.764)         &     (7.347)         &     (6.008)         &     (7.435)         \\
\hline
\(N\)       &       44772         &       44772         &       44772         &       44772         &       44772         \\
\hline\hline
\multicolumn{6}{l}{\footnotesize Standard errors in parentheses}\\
\multicolumn{6}{l}{\footnotesize \sym{*} \(p<0.05\), \sym{**} \(p<0.01\), \sym{***} \(p<0.001\)}\\
\end{tabular}
\end{table}


\textbf{2.1.B. Our additional controls}

Table \ref{tab:2.1B-deltacons-net}
\begin{table}[htbp]\centering
\def\sym#1{\ifmmode^{#1}\else\(^{#1}\)\fi}
\caption{\label{tab:2.1B-deltacons-net} Explaining change in consumption - net}
\begin{tabular}{l*{5}{c}}
\hline\hline
            &\multicolumn{1}{c}{(1)}         &\multicolumn{1}{c}{(2)}         &\multicolumn{1}{c}{(3)}         &\multicolumn{1}{c}{(4)}         &\multicolumn{1}{c}{(5)}         \\
\hline
avg-totcons-change&       0.996\sym{***}&                     &                     &                     &                     \\
            &    (0.0679)         &                     &                     &                     &                     \\
netinc-change&      0.0214\sym{***}&     0.00116\sym{***}&     0.00456\sym{***}&     0.00339\sym{***}&     0.00482\sym{***}\\
            &   (0.00137)         &  (0.000195)         &  (0.000499)         &  (0.000339)         &  (0.000526)         \\
age         &       3.029\sym{*}  &       0.186         &      -0.591         &      -0.682\sym{*}  &       0.360         \\
            &     (1.194)         &     (0.170)         &     (0.434)         &     (0.295)         &     (0.458)         \\
hhsize      &       48.43\sym{***}&       1.231         &       16.36\sym{***}&       15.00\sym{***}&      -20.32\sym{***}\\
            &     (13.33)         &     (1.899)         &     (4.852)         &     (3.299)         &     (5.117)         \\
1.quarter   &           0         &           0         &           0         &           0         &           0         \\
            &         (.)         &         (.)         &         (.)         &         (.)         &         (.)         \\
2.quarter   &      -8.535         &      -0.368         &      -1.240         &      -0.823         &      -3.310         \\
            &     (59.52)         &     (8.210)         &     (22.56)         &     (13.66)         &     (23.67)         \\
3.quarter   &      -7.273         &      -0.407         &      -1.313         &      -0.912         &      -2.234         \\
            &     (56.02)         &     (7.953)         &     (22.09)         &     (13.81)         &     (22.24)         \\
4.quarter   &       12.75         &       0.646         &       3.128         &       2.434         &       1.534         \\
            &     (55.25)         &     (7.834)         &     (23.26)         &     (13.47)         &     (23.45)         \\
year        &    -0.00331         &     0.00111         &     -0.0132         &     -0.0129         &      0.0109         \\
            &     (0.729)         &     (0.105)         &     (0.266)         &     (0.180)         &     (0.280)         \\
avg-food-change&                     &       1.002\sym{***}&                     &                     &                     \\
            &                     &    (0.0644)         &                     &                     &                     \\
avg-ndcons1-change&                     &                     &       1.000\sym{***}&                     &                     \\
            &                     &                     &    (0.0641)         &                     &                     \\
avg-ndcons2-change&                     &                     &                     &       0.999\sym{***}&                     \\
            &                     &                     &                     &    (0.0644)         &                     \\
avg-logndconsserv-change&                     &                     &                     &                     &       0.996\sym{***}\\
            &                     &                     &                     &                     &    (0.0659)         \\
constant    &      -290.9\sym{**} &      -13.35         &      -15.28         &      -6.402         &       32.94         \\
            &     (111.0)         &     (15.41)         &     (40.26)         &     (26.85)         &     (41.74)         \\
\hline
\(N\)       &       44772         &       44772         &       44772         &       44772         &       44772         \\
\hline\hline
\multicolumn{6}{l}{\footnotesize Standard errors in parentheses}\\
\multicolumn{6}{l}{\footnotesize \sym{*} \(p<0.05\), \sym{**} \(p<0.01\), \sym{***} \(p<0.001\)}\\
\end{tabular}
\end{table}


\textbf{2.1.C. Our additional controls (plus controlling for gross income changes instead of net income changes)}

Table \ref{tab:2.1B-deltacons-gross}
\begin{table}[!h]\centering
\def\sym#1{\ifmmode^{#1}\else\(^{#1}\)\fi}
\caption{\label{tab:2.1B-deltacons-gross} Explaining change in consumption - gross}
\begin{tabular}{l*{5}{c}}
\hline\hline
            &\multicolumn{1}{c}{(1)}         &\multicolumn{1}{c}{(2)}         &\multicolumn{1}{c}{(3)}         &\multicolumn{1}{c}{(4)}         &\multicolumn{1}{c}{(5)}         \\
\hline
grossinc-change&      0.0255\sym{***}&     0.00112\sym{***}&     0.00443\sym{***}&     0.00385\sym{***}&     0.00488\sym{***}\\
            &   (0.00137)         &  (0.000196)         &  (0.000500)         &  (0.000340)         &  (0.000527)         \\
age         &       3.158\sym{**} &       0.190         &      -0.576         &      -0.664\sym{*}  &       0.378         \\
            &     (1.192)         &     (0.170)         &     (0.434)         &     (0.295)         &     (0.458)         \\
hhsize      &       47.49\sym{***}&       1.238         &       16.39\sym{***}&       14.89\sym{***}&      -20.33\sym{***}\\
            &     (13.32)         &     (1.899)         &     (4.852)         &     (3.298)         &     (5.117)         \\
1.quarter   &           0         &           0         &           0         &           0         &           0         \\
            &         (.)         &         (.)         &         (.)         &         (.)         &         (.)         \\
2.quarter   &      -12.65         &      -0.435         &      -1.742         &      -1.178         &      -3.867         \\
            &     (59.45)         &     (8.210)         &     (22.56)         &     (13.66)         &     (23.67)         \\
3.quarter   &      -7.073         &      -0.296         &      -1.250         &      -0.713         &      -2.072         \\
            &     (55.95)         &     (7.953)         &     (22.10)         &     (13.80)         &     (22.24)         \\
4.quarter   &       14.31         &       0.643         &       2.578         &       2.723         &       1.209         \\
            &     (55.19)         &     (7.834)         &     (23.27)         &     (13.47)         &     (23.45)         \\
year        &     -0.0308         &   -0.000336         &     -0.0173         &     -0.0170         &     0.00629         \\
            &     (0.728)         &     (0.105)         &     (0.266)         &     (0.180)         &     (0.280)         \\
avg-totcons-change&       0.991\sym{***}&                     &                     &                     &                     \\
            &    (0.0679)         &                     &                     &                     &                     \\            
avg-food-change&                     &       1.001\sym{***}&                     &                     &                     \\
            &                     &    (0.0644)         &                     &                     &                     \\
avg-ndcons1-change&                     &                     &       0.998\sym{***}&                     &                     \\
            &                     &                     &    (0.0641)         &                     &                     \\
avg-ndcons2-change&                     &                     &                     &       0.996\sym{***}&                     \\
            &                     &                     &                     &    (0.0644)         &                     \\
avg-logndconsserv-change&                     &                     &                     &                     &       0.994\sym{***}\\
            &                     &                     &                     &                     &    (0.0659)         \\
constant    &      -293.7\sym{**} &      -13.45         &      -15.42         &      -6.911         &       32.51         \\
            &     (110.9)         &     (15.41)         &     (40.27)         &     (26.84)         &     (41.74)         \\
\hline
\(N\)       &       44772         &       44772         &       44772         &       44772         &       44772         \\
\hline\hline
\multicolumn{6}{l}{\footnotesize Standard errors in parentheses}\\
\multicolumn{6}{l}{\footnotesize \sym{*} \(p<0.05\), \sym{**} \(p<0.01\), \sym{***} \(p<0.001\)}\\
\end{tabular}
\end{table}


\textbf{2.1.D. We control for aggregate changes income (instead of aggregate changes in consumption)}

Idea here: Hypothesis 1 argues that individual consumption changes co-move one-to-one with the economy's average expenditure changes (and not with individual income changes). Besides proxying for changes in aggregate spending by changes in aggregate consumption (as we did before), we can proxy for it using changes in aggregate income. We do that below, controlling for changes in net/ gross individual income.

Table \ref{tab:2.1C-deltacons-net}
\begin{table}[!h]\centering
\def\sym#1{\ifmmode^{#1}\else\(^{#1}\)\fi}
\caption{\label{tab:2.1C-deltacons-net} Explaining change in consumption - net}
\begin{tabular}{l*{5}{c}}
\hline\hline
            &\multicolumn{1}{c}{(1)}         &\multicolumn{1}{c}{(2)}         &\multicolumn{1}{c}{(3)}         &\multicolumn{1}{c}{(4)}         &\multicolumn{1}{c}{(5)}         \\
\hline
avg-netinc-change&    -0.00414         &    -0.00654\sym{*}  &   -0.000709         &    -0.00187         &    -0.00364         \\
            &    (0.0223)         &   (0.00318)         &   (0.00811)         &   (0.00552)         &   (0.00856)         \\
netinc-change&      0.0215\sym{***}&     0.00116\sym{***}&     0.00458\sym{***}&     0.00340\sym{***}&     0.00484\sym{***}\\
            &   (0.00138)         &  (0.000196)         &  (0.000501)         &  (0.000341)         &  (0.000528)         \\
age         &       3.259\sym{**} &       0.211         &      -0.508         &      -0.647\sym{*}  &       0.463         \\
            &     (1.196)         &     (0.171)         &     (0.436)         &     (0.296)         &     (0.459)         \\
hhsize      &       47.82\sym{***}&       1.302         &       16.62\sym{***}&       15.35\sym{***}&      -20.64\sym{***}\\
            &     (13.36)         &     (1.904)         &     (4.865)         &     (3.308)         &     (5.130)         \\
1.quarter   &           0         &           0         &           0         &           0         &           0         \\
            &         (.)         &         (.)         &         (.)         &         (.)         &         (.)         \\
2.quarter   &      -333.8\sym{***}&      -34.45\sym{***}&      -161.2\sym{***}&      -4.796         &      -161.3\sym{***}\\
            &     (56.05)         &     (7.988)         &     (20.41)         &     (13.87)         &     (21.52)         \\
3.quarter   &      -107.0         &       12.82         &      -139.5\sym{***}&       18.17         &      -95.86\sym{***}\\
            &     (56.22)         &     (8.013)         &     (20.47)         &     (13.92)         &     (21.59)         \\
4.quarter   &      -137.0\sym{*}  &       15.58         &      -188.4\sym{***}&       15.25         &      -162.6\sym{***}\\
            &     (55.93)         &     (7.971)         &     (20.36)         &     (13.84)         &     (21.47)         \\
year        &     -0.0538         &      -0.267\sym{*}  &       0.195         &     -0.0173         &       0.265         \\
            &     (0.730)         &     (0.104)         &     (0.266)         &     (0.181)         &     (0.280)         \\
constant    &       99.82         &       10.01         &       125.9\sym{**} &       41.93         &       116.6\sym{**} \\
            &     (108.7)         &     (15.49)         &     (39.58)         &     (26.91)         &     (41.74)         \\
\hline
\(N\)       &       44772         &       44772         &       44772         &       44772         &       44772         \\
\hline\hline
\multicolumn{6}{l}{\footnotesize Standard errors in parentheses}\\
\multicolumn{6}{l}{\footnotesize \sym{*} \(p<0.05\), \sym{**} \(p<0.01\), \sym{***} \(p<0.001\)}\\
\end{tabular}
\end{table}


%%%%%%%%%%%%%%%%%%%%%%%%%%%%%%%%%%%%%%%%%%%%%%%%%%%%%%%%%%%%%%%%%%%%
\subsection*{Individual Consumption Growth depends only on Aggregate Consumption Growth}
What does the hypothesis expect? We would expect the coefficient on log.averagetotcons.change to be 1, and the coefficient on log.grossinc.change to be 0.

\textbf{2.2.A. Mace's specification}

Table \ref{tab:2.2A-logdeltacons}
\begin{table}[!h]\centering
\def\sym#1{\ifmmode^{#1}\else\(^{#1}\)\fi}
\caption{\label{tab:2.2A-logdeltacons} Explaining change in consumption growth}
\begin{tabular}{l*{5}{c}}
\hline\hline
            &\multicolumn{1}{c}{(1)}         &\multicolumn{1}{c}{(2)}         &\multicolumn{1}{c}{(3)}         &\multicolumn{1}{c}{(4)}         &\multicolumn{1}{c}{(5)}         \\
\hline
lognetinc-change&      0.0851\sym{***}&      0.0500\sym{***}&      0.0459\sym{***}&      0.0419\sym{***}&      0.0296\sym{***}\\
            &   (0.00360)         &   (0.00391)         &   (0.00293)         &   (0.00291)         &   (0.00240)         \\
avg-logtotcons-change&       0.996\sym{***}&                     &                     &                     &                     \\
            &    (0.0531)         &                     &                     &                     &                     \\
avg-logfood-change&                     &       1.018\sym{***}&                     &                     &                     \\
            &                     &    (0.0496)         &                     &                     &                     \\
avg-logndcons1-change&                     &                     &       1.002\sym{***}&                     &                     \\
            &                     &                     &    (0.0416)         &                     &                     \\
avg-logndcons2-change&                     &                     &                     &       0.996\sym{***}&                     \\
            &                     &                     &                     &    (0.0562)         &                     \\
avg-logndconsserv-change&                     &                     &                     &                     &       0.998\sym{***}\\
            &                     &                     &                     &                     &    (0.0448)         \\
constant    &    -0.00266         &    -0.00124         &    -0.00154         &    -0.00108         &   -0.000969         \\
            &   (0.00312)         &   (0.00248)         &   (0.00196)         &   (0.00239)         &   (0.00151)         \\
\hline
\(N\)       &       44448         &       44376         &       44443         &       44447         &       44448         \\
\hline\hline
\multicolumn{6}{l}{\footnotesize Standard errors in parentheses}\\
\multicolumn{6}{l}{\footnotesize \sym{*} \(p<0.05\), \sym{**} \(p<0.01\), \sym{***} \(p<0.001\)}\\
\end{tabular}
\end{table}


\textbf{2.2.B. Our additional controls}

Table \ref{tab:log2.2B-deltacons-net}
\begin{table}[htbp]\centering
\def\sym#1{\ifmmode^{#1}\else\(^{#1}\)\fi}
\caption{\label{tab:log2.2B-deltacons-net} Explaining change in consumption growth}
\begin{tabular}{l*{5}{c}}
\hline\hline
            &\multicolumn{1}{c}{(1)}         &\multicolumn{1}{c}{(2)}         &\multicolumn{1}{c}{(3)}         &\multicolumn{1}{c}{(4)}         &\multicolumn{1}{c}{(5)}         \\
\hline
avg-logtotcons-change&       0.989\sym{***}&                     &                     &                     &                     \\
            &    (0.0600)         &                     &                     &                     &                     \\
lognetinc-change&      0.0856\sym{***}&      0.0501\sym{***}&      0.0456\sym{***}&      0.0415\sym{***}&      0.0300\sym{***}\\
            &   (0.00360)         &   (0.00391)         &   (0.00294)         &   (0.00292)         &   (0.00240)         \\
age         &    0.000846\sym{***}&    0.000504\sym{***}&  -0.0000997         &   -0.000390\sym{***}&   0.0000351         \\
            &  (0.000140)         &  (0.000152)         &  (0.000114)         &  (0.000113)         & (0.0000930)         \\
hhsize      &     0.00412\sym{**} &     0.00520\sym{**} &     0.00475\sym{***}&     0.00386\sym{**} &    -0.00842\sym{***}\\
            &   (0.00156)         &   (0.00169)         &   (0.00127)         &   (0.00126)         &   (0.00104)         \\
1.quarter   &           0         &           0         &           0         &           0         &           0         \\
            &         (.)         &         (.)         &         (.)         &         (.)         &         (.)         \\
2.quarter   &    -0.00305         &    -0.00217         &    -0.00143         &    -0.00188         &    -0.00209         \\
            &   (0.00717)         &   (0.00732)         &   (0.00583)         &   (0.00524)         &   (0.00487)         \\
3.quarter   &    -0.00200         &   -0.000668         &   -0.000849         &   -0.000940         &    -0.00126         \\
            &   (0.00671)         &   (0.00709)         &   (0.00588)         &   (0.00530)         &   (0.00458)         \\
4.quarter   &     0.00231         &     0.00117         &     0.00205         &     0.00128         &  -0.0000257         \\
            &   (0.00691)         &   (0.00695)         &   (0.00662)         &   (0.00516)         &   (0.00506)         \\
year        &  0.00000873         & -0.00000570         & -0.00000720         & -0.00000802         &  0.00000384         \\
            & (0.0000853)         & (0.0000940)         & (0.0000695)         & (0.0000690)         & (0.0000568)         \\
avg-logfood-change&                     &       1.007\sym{***}&                     &                     &                     \\
            &                     &    (0.0556)         &                     &                     &                     \\
avg-logndcons1-change&                     &                     &       1.010\sym{***}&                     &                     \\
            &                     &                     &    (0.0532)         &                     &                     \\
avg-logndcons2-change&                     &                     &                     &       1.002\sym{***}&                     \\
            &                     &                     &                     &    (0.0571)         &                     \\
avg-logndconsserv-change&                     &                     &                     &                     &       0.990\sym{***}\\
            &                     &                     &                     &                     &    (0.0546)         \\
constant    &     -0.0559\sym{***}&     -0.0395\sym{**} &    -0.00861         &     0.00919         &      0.0201\sym{*}  \\
            &    (0.0133)         &    (0.0137)         &    (0.0107)         &    (0.0103)         &   (0.00860)         \\
\hline
\(N\)       &       44448         &       44376         &       44443         &       44447         &       44448         \\
\hline\hline
\multicolumn{6}{l}{\footnotesize Standard errors in parentheses}\\
\multicolumn{6}{l}{\footnotesize \sym{*} \(p<0.05\), \sym{**} \(p<0.01\), \sym{***} \(p<0.001\)}\\
\end{tabular}
\end{table}


\textbf{2.2.C. Our additional controls (plus controlling for gross income changes instead of net income changes)}

Here, we regress individual HH's consumption growth rates (in its different specifications) in the past 9 months (time between the surveys) on the growth rate of aggregate consumption of that type, the growth rate of the HH's income and HH characteristics such as (age, HH size) and seasonal variation. 

Table \ref{tab:log2.2B-deltacons-gross}
\begin{table}[htbp]\centering
\def\sym#1{\ifmmode^{#1}\else\(^{#1}\)\fi}
\caption{\label{tab:log2.2B-deltacons-gross} Explaining change in consumption growth - gross}
\begin{tabular}{l*{5}{c}}
\hline\hline
            &\multicolumn{1}{c}{(1)}         &\multicolumn{1}{c}{(2)}         &\multicolumn{1}{c}{(3)}         &\multicolumn{1}{c}{(4)}         &\multicolumn{1}{c}{(5)}         \\
\hline
avg-logtotcons-change&       0.987\sym{***}&                     &                     &                     &                     \\
            &    (0.0599)         &                     &                     &                     &                     \\
loggrossinc-change&      0.0873\sym{***}&      0.0502\sym{***}&      0.0455\sym{***}&      0.0419\sym{***}&      0.0318\sym{***}\\
            &   (0.00354)         &   (0.00384)         &   (0.00289)         &   (0.00287)         &   (0.00236)         \\
age         &    0.000854\sym{***}&    0.000503\sym{***}&  -0.0000981         &   -0.000398\sym{***}&   0.0000406         \\
            &  (0.000140)         &  (0.000151)         &  (0.000114)         &  (0.000113)         & (0.0000929)         \\
hhsize      &     0.00431\sym{**} &     0.00527\sym{**} &     0.00485\sym{***}&     0.00396\sym{**} &    -0.00828\sym{***}\\
            &   (0.00156)         &   (0.00169)         &   (0.00127)         &   (0.00126)         &   (0.00104)         \\
1.quarter   &           0         &           0         &           0         &           0         &           0         \\
            &         (.)         &         (.)         &         (.)         &         (.)         &         (.)         \\
2.quarter   &    -0.00305         &    -0.00141         &    -0.00122         &    -0.00151         &    -0.00168         \\
            &   (0.00716)         &   (0.00731)         &   (0.00582)         &   (0.00524)         &   (0.00486)         \\
3.quarter   &    -0.00139         &   -0.000182         &   -0.000393         &   -0.000358         &   -0.000664         \\
            &   (0.00670)         &   (0.00708)         &   (0.00587)         &   (0.00529)         &   (0.00457)         \\
4.quarter   &     0.00200         &     0.00176         &     0.00185         &     0.00139         &    0.000483         \\
            &   (0.00691)         &   (0.00694)         &   (0.00661)         &   (0.00515)         &   (0.00506)         \\
year        &   0.0000120         & -0.00000465         & -0.00000321         & -0.00000587         &  0.00000525         \\
            & (0.0000853)         & (0.0000939)         & (0.0000694)         & (0.0000690)         & (0.0000567)         \\
avg-logfood-change&                     &       1.004\sym{***}&                     &                     &                     \\
            &                     &    (0.0555)         &                     &                     &                     \\
avg-logndcons1-change&                     &                     &       1.005\sym{***}&                     &                     \\
            &                     &                     &    (0.0532)         &                     &                     \\
avg-logndcons2-change&                     &                     &                     &       1.000\sym{***}&                     \\
            &                     &                     &                     &    (0.0570)         &                     \\
avg-logndconsserv-change&                     &                     &                     &                     &       0.997\sym{***}\\
            &                     &                     &                     &                     &    (0.0545)         \\
constant    &     -0.0571\sym{***}&     -0.0405\sym{**} &    -0.00922         &     0.00880         &      0.0189\sym{*}  \\
            &    (0.0133)         &    (0.0137)         &    (0.0107)         &    (0.0103)         &   (0.00859)         \\
\hline
\(N\)       &       44646         &       44573         &       44641         &       44645         &       44646         \\
\hline\hline
\multicolumn{6}{l}{\footnotesize Standard errors in parentheses}\\
\multicolumn{6}{l}{\footnotesize \sym{*} \(p<0.05\), \sym{**} \(p<0.01\), \sym{***} \(p<0.001\)}\\
\end{tabular}
\end{table}


\end{document}

