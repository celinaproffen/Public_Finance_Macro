\documentclass[a4paper,12pt, parskip=half-]{scrartcl}

\usepackage{lmodern}
\usepackage[latin1, UTF8]{inputenc}
\usepackage[english]{babel}
\usepackage[T1]{fontenc}
\usepackage{amsmath}

% Define Fonts
\usepackage{helvet} %Arial
\renewcommand{\familydefault}{\sfdefault} %Arial
%\usepackage{mathptmx} %Times New Roman

\usepackage{graphicx}
\usepackage{graphics}
\usepackage[a4paper, top=2.6cm, bottom=2.6cm, right=2.7cm, left=2.7cm]{geometry}
\usepackage[onehalfspacing]{setspace}
\usepackage{microtype}
\usepackage{breakcites}
\usepackage{booktabs}
\usepackage{tabularx}
\usepackage{multirow} 
\pagenumbering{gobble}
\usepackage[absolute]{textpos}
\usepackage{wrapfig}
\usepackage{float}
\usepackage{placeins}
\usepackage{afterpage}
\usepackage{enumitem}
\usepackage{lscape}
\usepackage{caption}
\usepackage{subcaption}
\usepackage{listings}
\usepackage{multirow}
\usepackage{abstract}
\usepackage{tikz}
\setkomafont{sectioning}{\bfseries} 
\usepackage{rotating}
\usepackage{color}													
\usepackage{colortbl}
\usepackage{pdfpages}
\usepackage{listings}

\makeatletter
\newcommand{\MSonehalfspacing}{%
  \setstretch{1.44}%  default
  \ifcase \@ptsize \relax % 10pt
    \setstretch {1.448}%
  \or % 11pt
    \setstretch {1.399}%
  \or % 12pt
    \setstretch {1.433}%
  \fi
}
\makeatother
\MSonehalfspacing

\usepackage[authoryear,round]{natbib}
\bibliographystyle{apalike}

\deffootnote{0.6cm}{1em}{\makebox[0.6cm][l]{\thefootnotemark}}
\setkomafont{footnote}{\fontsize{10pt}{10pt}\selectfont}

\usepackage{fancyhdr}
\pagestyle{plain}

\definecolor{mygreen}{rgb}{0,0.6,0}
\definecolor{mygray}{rgb}{0.5,0.5,0.5}
\definecolor{mymauve}{rgb}{0.58,0,0.82}

\begin{document}

\textbf{Question 2: Summary of Cocco et al (2005)}\\

Cocco et al (2005) investigate a finite horizon life cycle model of consumption and portfolio choice with non-tradable labour income which can be seen as a substitute for risk-free asset holdings and market incompleteness, in particular borrowing constraints, using realistic restrictions. Therefore, they estimate the labour income stream for different educational groups using the PSID, and find that its shape induces the agent to decrease his (proportional) asset holdings over time. Further, agents who face higher income risk hold smaller portfolio shares in equities such that income risk crowds out risk from holding assets. The crowding out effect becomes larger when the model allows for disastrous labour income draws. Additionally, the authors investigate the effect of endogenous borrowing constraints in their incomplete market setting. They find that the lower bound for the income distribution is key in determining borrowing capacity and portfolio allocation. Agents whose income process is bounded are faced with a positive endogenous borrowing constraint that induces the young to hold negative wealth and do not invest in equities. Lastly, the authors compute the utility cost associated with portfolio decisions that are not optimal and find that there is a substantial utility loss occurring from investing a constant share - which would be optimal under complete markets - and ignoring the labour income stream.

\end{document}





