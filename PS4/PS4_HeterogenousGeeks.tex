\documentclass[12pt,a4paper]{article}

\usepackage[a4paper, top = 2cm, bottom = 2cm, left = 1.5cm, right = 1.5cm]{geometry}
\usepackage[dvipsnames]{xcolor} % Colors

\usepackage{standalone}

\usepackage{setspace}
\usepackage{graphicx}
\usepackage{amsfonts}
\usepackage{amsmath}
\usepackage{tikz}
\usepackage{pdfpages}
\usepackage{epigraph}
\usepackage{csquotes}
\usepackage{natbib}
% Bibliography
\usepackage{xcolor}
\usepackage{hyperref}
\hypersetup{
colorlinks=true,
citecolor=MidnightBlue,
linkcolor=MidnightBlue,
pdfpagemode=FullScreen}

\usepackage{natbib}
\usepackage[noabbrev]{cleveref}
\setcitestyle{authoryear,open={(},close={)}}
\bibliographystyle{plainnat}

\usepackage{subfiles}

\setlength\parindent{0pt}
\spacing{1.2}

\begin{document}

\begin{center}
       \vspace*{1cm}
       \huge\textbf{Project 4} \\
       \vspace{0.4cm}
       \large \textbf{Public Finance in Macroeconomics} \\
       \vspace{0.5cm}
        \large Handed in by the \textcolor{purple}{\textbf{Heterogeneous Geeks}} \\
        \vspace{0.3cm}
        a.k.a. Vivien Voigt, Thong Nguyen, \includegraphics[scale=0.06]{graphs/geek.png}\\Davide Difino \& Celina Proffen \\
       \vspace{1.5cm}
       \vfill



        Project in the context of Prof. Ludwig's course: \\
        \textbf{Public Finance in Macroeconomics: Heterogenous Agent Models}\\
        at the Graduate School of Economics, Finance and Management
       \vspace{0.8cm}
   \end{center}

\newpage

%%%%%%%%%%%%%%%%%%%%%%%%%%%%%%%%%%%%%%%%%%%%%%%%%%%%%%%%%%%%%%%%%%%%
%%%%%%%%%%%%%%%%%%%%%%%%%%%%%%%%%%%%%%%%%%%%%%%%%%%%%%%%%%%%%%%%%%%%

\section*{Problem 1: }

\includegraphics[width=0.5\textwidth]{PS4/graphs/smoothed_yy5b_hat.png}\\ \\
\includegraphics[width=0.5\textwidth]{PS4/graphs/smoothed_yy6b_hat.png}

\section*{Problem 2}
Please see code and remarks in code. We refer to the measures of income $yhh5$ and $yhh6$ for pre and post-government income respectively. As required, we deflate these measures and then apply logarithms before computing predicted values and residuals. 

\\

\section*{Problem 3: Interpretations}

\begin{table}[h]
\begin{tabular}{|l|l|l|}
\hline
\textbf{Variable} & \textbf{Pre government income} & \textbf{Post government income}  \\ \hline
$\rho$                        &   .9652454 (.0546179)  &  .8501573  (.038411)         \\ \hline
$\sigma^2_z$                  &   2.385936 (17.57936)  &  -1729.153 (3620.489)        \\ \hline
$\sigma^2_\nu$                &   .0550472 (.1076207)  &   .2028518 (.0676536)        \\ \hline
$\sigma^2_\epsilon$           &   .3610598  (.084534)  &   .1343607 (.0364851)        \\ \hline
$\sigma^2_{\Tilde{y}}$        &   1.167006             &   .8660633                   \\ \hline
\end{tabular}
\caption{\label{tab:estimates}The estimates for all variables but $\sigma^2_{\Tilde{y}}$ stem from estimations in the stata do.file handed in together with this project instructions. The estimate of $\sigma^2_{\Tilde{y}}$ is computed using the formula given in the project instructions and also derived below.}
\end{table}

Remember from the project instructions the following two equations
\begin{equation}\tag{1a}
    \Tilde{y}_{ijt} = z_{ijt} - \epsilon_{ijt}
\end{equation}
\begin{equation}\tag{1b}
    z_{ijt} = \rho z_{ij-1t-1} - \nu_{ijt}
\end{equation}

Let's start by deriving a new representation for residual income from (1) in the project instructions. First note that (1a) implies
\begin{align}
    z_{ij-1t-1} = \Tilde{y}_{ij-1t-1} - \epsilon_{ij-1t-1}
\end{align}
As $\epsilon$ is equally distributed over ages and individuals this can be rewritten as:
\begin{align}
    z_{ij-1t-1} = \Tilde{y}_{ij-1t-1} - \epsilon_{t-1}
\end{align}
Inserting the equation above in (1b) and noticing that $\ni$ is also distributed identically over ages and individuals:
\begin{align}
    z_{ijt} & = \rho(\Tilde{y}_{ij-1t-1} - \epsilon_{t-1}) + \nu_{ijt} \\
            & = \rho(\Tilde{y}_{ij-1t-1} - \epsilon_{t-1}) + \nu_{t}
\end{align}
Plugging in this transformation in (1a) yields:
\begin{align}
    \Tilde{y}_{ijt} = \rho\Tilde{y}_{ij-1t-1}  + \nu_{t} + \epsilon_t - \rho \epsilon_{t-1}
\end{align}

Now, for deriving the variance of $\Tilde{y}$, i.e. $\sigma^2_{\Tilde{y}}$, \textbf{a key assumption will be that all shocks are orthogonal to each other}, meaning that their co-variance is equal to zero. Also, it is important to assume that the process is weakly stationary (i.e. (co)variances do not change over time). 

Taking the variance of the equation above yields:
\begin{align}
    \sigma^2_{\Tilde{y}} & = \rho^2 \sigma^2_{\Tilde{y}} + \sigma^2_\nu + (1-\rho^2)\sigma^2_\epsilon \\
                         & = \frac{\sigma^2_\nu}{(1-\rho^2)} + \sigma^2_\epsilon
\end{align}
This proves the statement. The formula was used to estimate the variance of the residual income $\sigma^2_{\Tilde{y}}$.
\end{document}
