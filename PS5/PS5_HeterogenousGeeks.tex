\documentclass[12pt,a4paper]{article}

\usepackage[a4paper, top = 2cm, bottom = 2cm, left = 1.5cm, right = 1.5cm]{geometry}
\usepackage[dvipsnames]{xcolor} % Colors

\usepackage{standalone}

\usepackage{setspace}
\usepackage{graphicx}
\usepackage{amsfonts}
\usepackage{amsmath}
\usepackage{tikz}
\usepackage{pdfpages}
\usepackage{epigraph}
\usepackage{csquotes}
\usepackage{natbib}
\usepackage{accents}
% Bibliography
\usepackage{xcolor}
\usepackage{hyperref}
\hypersetup{
colorlinks=true,
citecolor=MidnightBlue,
linkcolor=MidnightBlue,
pdfpagemode=FullScreen}

\usepackage{listings}
\lstset{frame=tb,
  language=Matlab,
  aboveskip=3mm,
  belowskip=3mm,
  showstringspaces=false,
  columns=flexible,
  basicstyle={\small\ttfamily},
  numbers=none,
  numberstyle=\tiny\color{gray},
  keywordstyle=\color{Red},
  commentstyle=\color{MidnightBlue},
  stringstyle=\color{mauve},
  breaklines=true,
  breakatwhitespace=true,
  tabsize=3
}

\usepackage{natbib}
\usepackage[noabbrev]{cleveref}
\setcitestyle{authoryear,open={(},close={)}}
\bibliographystyle{plainnat}

\usepackage{subfiles}

\usepackage{url}
\urlstyle{same} % omit this command if you want monospaced-font look
\newcommand\purl[1]{\protect\url{#1}} % "protected url"

\setlength\parindent{0pt}
\spacing{1.2}

\begin{document}

\begin{center}
       \vspace*{4cm}
       \huge\textbf{Project 5} \\
       \vspace{0.4cm}
       \large \textbf{Public Finance in Macroeconomics} \\
       \vspace{0.5cm}
        \large Handed in by the \textcolor{orange}{\textbf{Heterogeneous Geeks}} \\
        \vspace{0.3cm}
        a.k.a. Vivien Voigt, Thong Nguyen, \includegraphics[scale=0.06]{geek.png}\\Davide Difino \& Celina Proffen \\
       \vspace{1.5cm}
       \vfill



        Project in the context of Prof. Ludwig's course: \\
        \textbf{Public Finance in Macroeconomics: Heterogenous Agent Models}\\
        at the Graduate School of Economics, Finance, and Management
       \vspace{0.8cm}
   \end{center}

\newpage

%-----------------------------------------------------

\section*{Problem 1: Solution}

\subsection*{Explain all sections of the code}

The main function presented in {\fontfamily{ccr}\selectfont towards\_olg} are:

\begin{lstlisting}[frame=single]
% -------------------------------------------------------------------------------%
% use all functions below to print graphs
function towards_olg
% -------------------------------------------------------------------------------%

% -------------------------------------------------------------------------------%
% define the values of parameters
function func_calibr(opt_det,opt_nosr,opt_ny)
% ------------------------------------------------------------------------------- %

% ------------------------------------------------------------------------------- %
% solution of the household problem
%    prints:
%       - state variable matrix (gridx)
%       - saving grid (gridsav)
%       - optimal consumption function (cfun)
%       - optimal value function (vfun)
function [gridx,gridsav,gridass,cfun,vfun] = func_hh
% ------------------------------------------------------------------------------- %

% ------------------------------------------------------------------------------- %
% aggregation and cross-sectional measure
%   prints:
%       - distribution of assets conditional by age and shock (Phi)
%       - distribution of assets (PhiAss)
function [Phi,PhiAss,ass]=func_aggr(gridx,gridsav,cfun,gridass)
% ------------------------------------------------------------------------------- %

% ------------------------------------------------------------------------------- %
% compute average life-cycle profiles
function [labinclife,inclife,asslife,conslife,vallife] = lcprofile(Phi,gridass,cfun,vfun)
% ------------------------------------------------------------------------------- %

% ------------------------------------------------------------------------------- %
% utility function
function u = U(c)
% ------------------------------------------------------------------------------- %
\end{lstlisting}
\pagebreak
\begin{lstlisting}[frame=single]
% ------------------------------------------------------------------------------- %
% marginal utility function
function muc=MUc(c)
% ------------------------------------------------------------------------------- %

% ------------------------------------------------------------------------------- %
% construction of the Markov chain process
%   prints:
%       - transition matrix first eigenvector (pini)
%       - shocks values (gridy)
%       - transition matrix (pi)
function [pini,pi,gridy]=mchain(rhoeta,epsil)
% ------------------------------------------------------------------------------- %

% ------------------------------------------------------------------------------- %
% this function computes the inverted of the utility function
% it is required in func_hh to retrive optimal consumption from the value function
function invut=invut(marg)
% ------------------------------------------------------------------------------- %

% ------------------------------------------------------------------------------- %
% create curved grid using curvature parameter c
function grd = makegrid(x1,x2,n,c)
% ------------------------------------------------------------------------------- %
\end{lstlisting}

For a more detailed description, please refer to {\fontfamily{ccr}\selectfont toward\_olg\_annotated}.

\subsection*{Compare the solution procedure in {\fontfamily{ccr}\selectfont func\_hh} to a standard implementation of the exogenous grid method}

The solution procedure in {\fontfamily{ccr}\selectfont func\_hh} uses the "endogenous gridpoints approach" for solving dynamic stochastic optimization problems, as introduced by \citeauthor{carroll2002lecture} (\citeyear{carroll2002lecture}, \citeyear{carroll2006method}). Instead of creating using an "exogenous" - in that it is given by the coder - grid for the state variable cash-on-hand ($x$), the "endogenous" approach exploit the definion

$$ s = x - c $$

- that is, (end of period) savings are equal to cash-on-hand minus consumption - to compute a new cash-on-hand grid in each period of iteration (over time and possible shock realization), given a fixed (and "exogenous") grid for $s$. Furthermore, substituting $s$ into the Euler equation allows to solve easily for consumption for each point of the saving grid. That is, for each iteration, a pair $s$ and $c(s)$ is defined. The endogenous cash-on-hand grid is then $x(s) = s + c(s)$, and represent the level of cash-on-hand that are rquired for each level of end-of-period savings (and optimal consumption given saving target). Then the value function in current period is updated for each value of the new endogenous cash-on-hand grid (given age and shock, still) by interpolation.

The exogenous grid approach, on the other hand, does both the derivation of the conditional policy function and the updating of the value function at the end of the iteration on the endogenous grid.

The advantages of the exogenous grid method are twofold: first of all, it does not require a nonlinear rootfinding since the Euler equation is solved for $c$ given $s$, instead of $x$, thus removing consumption from the RHS. Second, updating the value function over the endogenous grid for $x$ is more efficient in that the value function ($V'$) is interpolated only for values of $x'$ that have been used to compute $c'$ in the first istance, while the exogenous approch loops over \textit{all} possible $x'$.



\subsection*{Implement the standard method of exogenous grid points by working with an exogenous grid for cash-on-hand x}

\subsection*{Implement a standard value function iteration}

\subsection*{Look at funcaggr}

\subsection*{Solve the model under the following three alternative settings:}


\section*{Problem 2: Calibration}

\section*{Problem 3: Analysis}

\section*{Problem 4: Summary of \cite{de2004wealth}}
Since the distribution of wealth is much more concentrated than the distribution of (labour) earnings, De Nardi (2004) estimates a quantitative, general equilibrium, incomplete markets, overlapping-generations model to match this observation from the data. In doing so, the author links parents and children by accidental as well as voluntary bequests - but not by inter vivo transfers - and allows children to inherit some of their parents' productivity. Further, she calibrates the model to match important features of, first, the data of the U.S. and, second, the data of Sweden. Both countries have different wealth distribitions even though their Gini coefficients are relatively similar. This paper's framework makes it possible for intergenerational links to induce saving behaviour that generates a more concentrated distribution of wealth than that of earnings due to allowing two different saving motives: self-insurance against labour earnings shocks and life-span risk, i.e. saving for retirement, which both can be seen as precautionary motives, and the preference to leave bequests to their offspring, which is an altruistic motive.
\\
De Nardi finds that saving for precautionary purposes and saving for retirement are the primary factors for wealth accumulation at the lower tail of the distribution, while saving to leave bequests significantly affects the shape of the upper tail.

\pagebreak

\bibliography{Literature/bibliography.bib}

\end{document}
