\documentclass[12pt,a4paper]{article}

\usepackage[a4paper, top = 2cm, bottom = 2cm, left = 1.5cm, right = 1.5cm]{geometry}
\usepackage[dvipsnames]{xcolor} % Colors

\usepackage{standalone}

\usepackage{setspace}
\usepackage{graphicx}
\usepackage{amsfonts}
\usepackage{amsmath}
\usepackage{tikz}
\usepackage{pdfpages}
\usepackage{epigraph}
\usepackage{csquotes}
\usepackage{natbib}
\usepackage{accents}
% Bibliography
\usepackage{xcolor}
\usepackage{hyperref}
\hypersetup{
colorlinks=true,
citecolor=MidnightBlue,
linkcolor=MidnightBlue,
pdfpagemode=FullScreen}

\usepackage{listings}
\lstset{frame=tb,
  language=Matlab,
  aboveskip=3mm,
  belowskip=3mm,
  showstringspaces=false,
  columns=flexible,
  basicstyle={\small\ttfamily},
  numbers=none,
  numberstyle=\tiny\color{gray},
  keywordstyle=\color{Red},
  commentstyle=\color{MidnightBlue},
  stringstyle=\color{mauve},
  breaklines=true,
  breakatwhitespace=true,
  tabsize=3
}

\usepackage{natbib}
\usepackage[noabbrev]{cleveref}
\setcitestyle{authoryear,open={(},close={)}}
\bibliographystyle{plainnat}

\usepackage{subfiles}

\usepackage{url}
\urlstyle{same} % omit this command if you want monospaced-font look
\newcommand\purl[1]{\protect\url{#1}} % "protected url"

\setlength\parindent{0pt}
\spacing{1.2}

\begin{document}

\begin{center}
       \vspace*{4cm}
       \huge\textbf{Project 5} \\
       \vspace{0.4cm}
       \large \textbf{Public Finance in Macroeconomics} \\
       \vspace{0.5cm}
        \large Handed in by the \textcolor{orange}{\textbf{Heterogeneous Geeks}} \\
        \vspace{0.3cm}
        a.k.a. Vivien Voigt, Thong Nguyen, \includegraphics[scale=0.06]{geek.png}\\Davide Difino \& Celina Proffen \\
       \vspace{1.5cm}
       \vfill



        Project in the context of Prof. Ludwig's course: \\
        \textbf{Public Finance in Macroeconomics: Heterogenous Agent Models}\\
        at the Graduate School of Economics, Finance, and Management
       \vspace{0.8cm}
   \end{center}

\newpage

%-----------------------------------------------------

\section*{Problem 1: Solution}

\subsection*{Explain all sections of the code}

The main function presented in {\fontfamily{ccr}\selectfont towards\_olg} are:

\begin{lstlisting}[frame=single]
% -------------------------------------------------------------------------------%
% use all functions below to print graphs
function towards_olg
% -------------------------------------------------------------------------------%

% -------------------------------------------------------------------------------%
% define the values of parameters
function func_calibr(opt_det,opt_nosr,opt_ny)
% ------------------------------------------------------------------------------- %

% ------------------------------------------------------------------------------- %
% solution of the household problem
%    prints:
%       - state variable matrix (gridx)
%       - saving grid (gridsav)
%       - optimal consumption function (cfun)
%       - optimal value function (vfun)
function [gridx,gridsav,gridass,cfun,vfun] = func_hh
% ------------------------------------------------------------------------------- %

% ------------------------------------------------------------------------------- %
% aggregation and cross-sectional measure
%   prints:
%       - distribution of assets conditional by age and shock (Phi)
%       - distribution of assets (PhiAss)
function [Phi,PhiAss,ass]=func_aggr(gridx,gridsav,cfun,gridass)
% ------------------------------------------------------------------------------- %

% ------------------------------------------------------------------------------- %
% compute average life-cycle profiles
function [labinclife,inclife,asslife,conslife,vallife] = lcprofile(Phi,gridass,cfun,vfun)
% ------------------------------------------------------------------------------- %

% ------------------------------------------------------------------------------- %
% utility function
function u = U(c)
% ------------------------------------------------------------------------------- %
\end{lstlisting}
\pagebreak
\begin{lstlisting}[frame=single]
% ------------------------------------------------------------------------------- %
% marginal utility function
function muc=MUc(c)
% ------------------------------------------------------------------------------- %

% ------------------------------------------------------------------------------- %
% construction of the Markov chain process
%   prints:
%       - transition matrix first eigenvector (pini)
%       - shocks values (gridy)
%       - transition matrix (pi)
function [pini,pi,gridy]=mchain(rhoeta,epsil)
% ------------------------------------------------------------------------------- %

% ------------------------------------------------------------------------------- %
% this function computes the inverted of the utility function
% it is required in func_hh to retrive optimal consumption from the value function
function invut=invut(marg)
% ------------------------------------------------------------------------------- %

% ------------------------------------------------------------------------------- %
% create curved grid using curvature parameter c
function grd = makegrid(x1,x2,n,c)
% ------------------------------------------------------------------------------- %
\end{lstlisting}

For a more detailed description, please refer to {\fontfamily{ccr}\selectfont toward\_olg\_annotated}.

\subsection*{Compare the solution procedure in {\fontfamily{ccr}\selectfont func\_hh} to a standard implementation of the exogenous grid method}

The solution procedure in {\fontfamily{ccr}\selectfont func\_hh} uses the "endogenous gridpoints approach" for solving dynamic stochastic optimization problems, as introduced by \citeauthor{carroll2002lecture} (\citeyear{carroll2002lecture}, \citeyear{carroll2006method}). Instead of creating using an "exogenous" - in that it is given by the coder - grid for the state variable cash-on-hand ($x$), the "endogenous" approach exploit the definion

$$ s = x - c $$

- that is, (end of period) savings are equal to cash-on-hand minus consumption - to compute a new cash-on-hand grid in each period of iteration (over time and possible shock realization), given a fixed (and "exogenous") grid for $s$. Furthermore, substituting $s$ into the Euler equation allows to solve easily for consumption for each point of the saving grid. That is, for each iteration, a pair $s$ and $c(s)$ is defined. The endogenous cash-on-hand grid is then $x(s) = s + c(s)$, and represent the level of cash-on-hand that are rquired for each level of end-of-period savings (and optimal consumption given saving target). Then the value function in current period is updated for each value of the new endogenous cash-on-hand grid (given age and shock, still) by interpolation.

The exogenous grid approach, on the other hand, does both the derivation of the conditional policy function and the updating of the value function at the end of the iteration on the endogenous grid.

The advantages of the exogenous grid method are twofold: first of all, it does not require a nonlinear rootfinding since the Euler equation is solved for $c$ given $s$, instead of $x$, thus removing consumption from the RHS. Second, updating the value function over the endogenous grid for $x$ is more efficient in that the value function ($V'$) is interpolated only for values of $x'$ that have been used to compute $c'$ in the first istance, while the exogenous approch loops over \textit{all} possible $x'$.



\subsection*{Implement the standard method of exogenous grid points by working with an exogenous grid for cash-on-hand x}

\subsection*{Implement a standard value function iteration}

\subsection*{Look at funcaggr}

\subsection*{Comparison of Models}
\textbf{Comparison of Models}

In the attached PDF (Comparison model outcomes) I plotted all required scenarios in a way that (hopefully) allows an easier comparison\footnote{The asset distribution at different ages was cut down to just display it at age 80 out of space constraints.}. I have classified the models as follows, whereby the red coloured part identifies the deviations from the standard model:

\begin{itemize}
    \item Standard: opt\_det = false, opt\_nosr = true, tetta=2, r=0.04, rho=0.04
    \item Variation 1: opt\_det = true, opt\_nosr = true, tetta=2, r=0.04, rho=0.04
    \item Variation 2: \textcolor{red}{opt\_det = true, opt\_nosr = false}, tetta=2, r=0.04, rho=0.04
    \item Variation 3: opt\_det = false, \textcolor{red}{opt\_nosr = false}, tetta=2, r=0.04, rho=0.04
    \item Variation 4: opt\_det = false, opt\_nosr = true, \textcolor{red}{tetta=1}, r=0.04, rho=0.04
    \item Variation 5: opt\_det = false, opt\_nosr = true, \textcolor{red}{tetta=5}, r=0.04, rho=0.04
    \item Variation 6: opt\_det = false, opt\_nosr = true, tetta=2, r=0.04, \textcolor{red}{rho=0.4} 
    \item Variation 7: opt\_det = false, opt\_nosr = true, tetta=2, \textcolor{red}{r=0.4}, rho=0.04 
\end{itemize}

\textbf{The standard model} interest rates and impatience parameter balance each other out, s.t. the left and RHS of the standard Euler equation (w.o. accounting for possible borrowing constraints) simplify to the marginal utility of consumption today having to equal the expected marginal utility of consumption tomorrow. However, there is a precautionary motive turned on given the positive third derivative (presence of prudence) and variability in the income process (it is NOT deterministic but rather modelled by a Markov process). People know for certain when they will die which explains the shape of assets that monotonically approaches zero from above (last part of the asset function over the life cycle is concave). the intuition is, that people do not want to die with left-over assets as they still derive utility from them. \\

\textbf{Variation 1} makes income deterministic, which will turn of the precautionary savings behavior. The distribution of assets is suddenly more dense, i.e. people do not behave so differently in their savings behavior, as they do not experience different income states. They also build up the assets more slowly than they did under the standard specifications. Fianlly, it is to note that there is a massive change in the consumption profile of agents. Looking at the scale that corresponds to the consumption level graph, the changes in absolute consumption are smaller than those in the standard model. The consume their income almost fully until they are about to retire. In the last years of the working life the agents start saving a larger portion (as they become concerned with their retirement time) and during the retirement their consumption remains constant (as it did in the standard model). \\

\textbf{Variation 3} (I will come to Variation 2 next), uses the standard model but now changes besides having stochastic income allows for survival risk (i.e. people never know when they will exactly die). A 100 is however the moment, when one dies with absolute certainty. This survival risk is another source of risk, which can change the consumption pattern. As in the standard model, the asset distribution is rather wide, although a bit lower than in the standard case (people apparently don't want to save as much as before, as they never know whether this means having wasted consumption during the precious time alive ;) Accordingly savings are also a bit lower, and consumption a bit higher than in the standard case. In line with the logic described above, the shape of the consumption function over the life cycle decreases over age (and does not stay constant and high as in the standard model w/o survival risk). \\

\textbf{Variation 2} combines survival risk and a deterministic income process: i.e. we have more risk of dying unpredictably, but save earnings. Again, like in variation 1, the distribution of assets is much more narrow, but this time focuses even more at an early age (when it is still pretty unlikely that one will die).  In comparison to Variation 1, which has no survival risk, we see that the consumption profile is hump-shaped, in line with the idea that one knows that it is ever more unlikely that one will survive and doe snot want to leave too many assets behind without consuming them. In contrast to Variation 3, where income is not deterministic, people only start saving more out of current income as they approach the retirement age. This is in line with the idea, that the labor income during working age (conditional on surviving) will come in in its full/ constant amount for sure.

\textbf{Variation 4 and 5}. Looking at Variations 4 and 5, we can make some observations on how tetta, the coefficient of relative risk aversion, changes consumption/ savings behavior in a model of stochastic income but with no survival risk. A higher tetta implies that the RHS of the Euler equation will increase with uncertainty, and that we will want to save more. This is hard to see in the graphs, but there is indeed some differences over the assets accumulated over the life cycle, when one pays close attention to the savings and the assets graph. Also, and perhaps more clearly, the graph on the consumption function over the life cycle reveals that during the retirement age, people with a higher tetta are able to consume more (given their savings).\\
Also note, that 1/tetta= IES, the intertemporal elasticity of substitution. The latter makes people more willing to adapt their consumption intertemporally when they see an important gain (e.g. in interest rates) from doing so. Here, the interest rate and the intertemporal discount factor equal each other out, but we can still see that people with low IES (high tetta) have a somewhat wider asset distribution over the life cycle, while the ones with low tetta (high IES) take advantage of the compounded interest effects, i.e. they focus the distribution of assets more on the younger ages to take more advantage of it later. 

\textbf{Variation 6 and 7} show how the patterns change when (given a constant tetta) the relationship between interest rates and discount factor change. Call $A = \frac{1+r}{1+\rho}$; a high $A$ should lead to more savings today (i.e. lower consumption, higher marginal utility from consumption today). Variation 6 entails a very low $A$ and indeed has relatively low savings when comparing it to the Standard case. This is exactly opposite in the Variation 7, where savings reach over 20 times the level of variation 6. Instead of the humped shaped consumption that follows for Variation 6, in 7 we observe a steady and almost exponential increase in consumption of the life cycle. From a highly negative value function in the early years, this rises to significantly higher levels than those observed in variation 6 at approx. the age of 30. 

\textcolor{red}{@someone: Please note that the labor income in Variation 7 looks really odd- maybe I chose parameters that are not consistent at some point. I still think we can use these graphs here, but I would recommend explaining/ understanding why this happened. }



\end{document}


\section*{Problem 2: Calibration}

\section*{Problem 3: Analysis}

\section*{Problem 4: Summary of \cite{de2004wealth}}
Since the distribution of wealth is much more concentrated than the distribution of (labour) earnings, De Nardi (2004) estimates a quantitative, general equilibrium, incomplete markets, overlapping-generations model to match this observation from the data. In doing so, the author links parents and children by accidental as well as voluntary bequests - but not by inter vivo transfers - and allows children to inherit some of their parents' productivity. Further, she calibrates the model to match important features of, first, the data of the U.S. and, second, the data of Sweden. Both countries have different wealth distribitions even though their Gini coefficients are relatively similar. This paper's framework makes it possible for intergenerational links to induce saving behaviour that generates a more concentrated distribution of wealth than that of earnings due to allowing two different saving motives: self-insurance against labour earnings shocks and life-span risk, i.e. saving for retirement, which both can be seen as precautionary motives, and the preference to leave bequests to their offspring, which is an altruistic motive.
\\
De Nardi finds that saving for precautionary purposes and saving for retirement are the primary factors for wealth accumulation at the lower tail of the distribution, while saving to leave bequests significantly affects the shape of the upper tail.

\pagebreak

\bibliography{Literature/bibliography.bib}

\end{document}
