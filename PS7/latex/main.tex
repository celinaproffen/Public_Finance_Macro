\documentclass[12pt,a4paper]{article}

\usepackage[a4paper, top = 2cm, bottom = 2cm, left = 1.5cm, right = 1.5cm]{geometry}
\usepackage[dvipsnames]{xcolor} % Colors

\usepackage{standalone}
\usepackage{amsmath} 
\usepackage{setspace}
\usepackage{graphicx}
\usepackage{amsfonts}
\usepackage{amsmath}
\usepackage{tikz}
\usepackage{pdfpages}
\usepackage{epigraph}
\usepackage{csquotes}
\usepackage{natbib}
\usepackage{accents}
\usepackage{pdfpages}

% Bibliography
\usepackage{xcolor}
\usepackage{hyperref}
\hypersetup{
colorlinks=true,
citecolor=MidnightBlue,
linkcolor=MidnightBlue,
pdfpagemode=FullScreen}

\usepackage{listings}
\lstset{frame=tb,
  language=Matlab,
  aboveskip=3mm,
  belowskip=3mm,
  showstringspaces=false,
  columns=flexible,
  basicstyle={\small\ttfamily},
  numbers=none,
  numberstyle=\tiny\color{gray},
  keywordstyle=\color{Red},
  commentstyle=\color{MidnightBlue},
  stringstyle=\color{Red},
  breaklines=true,
  breakatwhitespace=true,
  tabsize=3
}

\usepackage{natbib}
\usepackage[noabbrev]{cleveref}
\setcitestyle{authoryear,open={(},close={)}}
\bibliographystyle{plainnat}

\usepackage{subfiles}

\usepackage{url}
\urlstyle{same} % omit this command if you want monospaced-font look
\newcommand\purl[1]{\protect\url{#1}} % "protected url"

\setlength\parindent{0pt}
\spacing{1.2}

\begin{document}

\begin{center}
       \vspace*{4cm}
       \huge\textbf{Project 7} \\
       \vspace{0.4cm}
       \large \textbf{Public Finance in Macroeconomics} \\
       \vspace{0.5cm}
        \large Handed in by the \textcolor{orange}{\textbf{Heterogeneous Geeks}} \\
        \vspace{0.3cm}
        a.k.a. Vivien Voigt, Thong Nguyen, \includegraphics[scale=0.06]{geek.png}\\Davide Difino \& Celina Proffen \\
       \vspace{1.5cm}
       \vfill



        Project in the context of Prof. Ludwig's course: \\
        \textbf{Public Finance in Macroeconomics: Heterogenous Agent Models}\\
        at the Graduate School of Economics, Finance, and Management
       \vspace{0.8cm}
   \end{center}

\newpage

\section{Simple Variant of Krusell-Smith Algorithm}

\subsection{Characterization of Equilibrium Dynamics}

The proof of Proposition 3 uses a Guess and Verify approach:

\textcolor{red}{@Davide: Are you sure about lambda beinf the growth rate of labor "PORDUCTIVITY"? Isn't it just labor L = 1+ lambda}

\begin{enumerate}

  \item Guess that households will save a constant share $s$ of (disposable) wage,

    \begin{equation}
      a_{2,t+1} = s(1-\tau)w_t
    \label{guess}
    \end{equation}

  Since the firm optimization problem

    \[
      \max \Pi = \zeta_t F(K_t, \Upsilon_t L) - (\bar{\delta} + r_t)\varrho_t^{-1} K_t - w_t L,
    \]

  where $\zeta_t$ and $\varrho_t$ are aggregate shocks, imply the f.o.c.

    \[
      w_{t} = (1-\alpha)\Upsilon_t k_t^\alpha \zeta_t
    \]

  where

    \begin{equation}
        k_t = \frac{K_t}{\Upsilon_t L},
        \label{K_per_L}
    \end{equation}

  then \ref{guess} becomes

    \begin{equation}
        a_{2,t+1} = s(1-\tau)(1-\alpha)\Upsilon_t k_t^\alpha \zeta_t
    \label{a_guess}
    \end{equation}

  Market clearing conditions $K_t = a_{2,t}$ and $L = 1 + \lambda$, growth rate of labor productivity $\Upsilon_{t+1} = (1+g)\Upsilon_{t}$  and \ref{K_per_L} imply that

    \begin{equation}
        k_{t+1} = \frac{s(1-\tau)(1-\alpha) k_t^\alpha \zeta_t}{(1 + g)(1 + \lambda)}
        \label{K_guess}
    \end{equation}

  which is the equilibrium dynamics implied by our guess $s$.

  \item Under the market clearing condition We know that, given $s$, \ref{K_guess} also solves the Firm's problem. Then to verify that our guess indeed an equilibrium solution we need to check for which value of $s$ the household problem is solved.

  The agent budget constraints, substituting for $w_t$, $w_{t + 1}$ and $\alpha_{2, t+1}$ using \ref{a_guess} implies consumption levels

    \begin{align*}
      c_{1,t}& = (1 - s) w_t =(1 - s)(1 - \alpha) \Upsilon_t k_t^\alpha \zeta_t  \\
      c_{2,t+1} & = s(1 - \tau) (1 - \alpha)\zeta_{t} \Upsilon_t k_t^\alpha \alpha \zeta_{t+1} \varrho_{t+1} k_{t+1}^{\alpha - 1}   \\
      & + (1 - \alpha) \Upsilon_{t+1}\zeta_{t+1} k_{t+1}^{\alpha} (\lambda \eta_{2, t+1} + \tau (1 + \lambda(1 - \eta_{2, t+1})))
    \end{align*}

  in the two periods lived by the agent, where the interest rate is provided by the second f.o.c of the firm,

  \[
    1 + r_t = \alpha k_t^{\alpha - 1} \zeta_{t} \varrho_{t}
  \]

  Substituting \ref{K_guess} yields

    \begin{align*}
     c_{2,t+1} & = (\alpha \varrho_{t+1} (1 + \lambda) + (1 + \alpha)(\lambda \eta_{2, t+1}  \\
     & + \tau(1 + \lambda(1 - \eta_{2, t+1})))) \Upsilon_{t+1}\zeta_{t+1} k_{t+1}^{\alpha}.
    \end{align*}

  Substituting the consumption levels $c_{1,t}$ and $c_{2,t+1}$ into the consumer Euler Equation

    \[
      1 = \beta \mathbb{E}_t \left[ \frac{c_{1,t} (1+r_{t+1})}{c_{2,t+1}} \right]
    \]

  implies

    \[
    1 =  \frac{\beta(1 - s)}{s}\Phi(\tau)
    \]

  where $\Phi(\tau)$ is a function of pension system contribution $\tau$ as defined in Proposition 3. Therefore, optimal saving rate in general equilibrium is also a function of $\tau$:

    \begin{equation}
      s(\tau) = \frac{\beta \Phi(\tau)}{1 + \beta \Phi(\tau)}.
      \label{opt_s}
    \end{equation}

  \item Due to convexity, the solution represented by \ref{K_guess} and \ref{opt_s} is unique.  Q.e.d.

\end{enumerate}

\newpage
\subsection{Discretize shocks}
\vspace{1cm}
\begin{equation*}
    \zeta_-=\frac{2exp(1-\sigma_{ln\zeta})}{exp(1-\sigma_{ln\zeta})+exp(1+\sigma_{ln\zeta})}\approx\frac{2exp(1-0.13)}{exp(1-0.13)+exp(1+0.13)}\approx 0.87
    \end{equation*}
    \begin{equation*}
       \zeta_+=\frac{2exp(1+\sigma_{ln\zeta})}{exp(1-\sigma_{ln\zeta})+exp(1+\sigma_{ln\zeta})}\approx\frac{2exp(1+0.13)}{exp(1-0.13)+exp(1+0.13)}\approx 1.13
    \end{equation*}

\vspace{1cm}
\begin{equation*}
    \varrho_-=\frac{2exp(1-\simga_{ln\varrho})}{exp(1-\simga_{ln\varrho})+exp(1+\simga_{ln\varrho})}\approx\frac{2exp(1-0.5)}{exp(1-0.5)+exp(1+0.5)}\approx 0.54
    \end{equation*}
    \begin{equation*}
       \varrho_+=\frac{2exp(1+\simga_{ln\varrho})}{exp(1-\simga_{ln\varrho})+exp(1+\simga_{ln\varrho})}\approx\frac{2exp(1+0.5)}{exp(1-0.5)+exp(1+0.5)}\approx 1.46
    \end{equation*}

\vspace{1cm}
Using the hint that ln($\eta$) is normally distributed, we receive the following shock distribution.
\begin{center}
    \includegraphics[width=15cm]{eta_shocks.png}\\
\end{center}
Assuming -4 and 3 as the bounds of the distribution, we receive the following 11 nodes for $\eta$ using the Gaussian Quadratiture method: \\

$\eta_1 = -4.0000  $ \\
$\eta_2 = -3.2192 $ \\
$\eta_3 = -2.6497 $ \\
$\eta_4 = -1.8838 $ \\
$\eta_5 = -0.9782 $ \\
$\eta_6 = -0.4500 $ \\
$\eta_7 = 0.9506 $ \\
$\eta_8 = 1.8307 $ \\
$\eta_9 = 2.5751 $ \\
$\eta_{10} = 3.1285 $ \\
$\eta_{11} = 3.0000 $ \\

\subsection{Simulation of capital dynamics}
Please refer to Matlab code.

\subsection{Krusell-Smith Algorithm}
Please also refer to the matlab code. I will outline the procedure for solving step 3.(b).i. here, i.e. solving for savings rates $s_t$ as a function of $\tau$,$k_t$ and $z$.

This basically relies again on the Euler equation:
\begin{align*}
          1 = \beta \mathbb{E}_t \left[ \frac{c_{1,t} (1+r_{t+1})}{c_{2,t+1}} \right]\\
          \mathbb{E}_t\left[log\left(c_{2,t+1}\right)\right] = log\left(\beta\right) + log\left( \mathbb{E}_t \left[1+r_{t+1}\right]\right)  + log\left( c_{1,t} \right)
\end{align*}

We can use the expression from (1)-(3) in Harenberg-Ludwig (2015) to insert the expressions.
\pagebreak
\begin{align*}
          log\left( \mathbb{E}_t\left[ s(1-\tau)(1+r_{t+1}) + \lambda\eta_{2,t+1}(1-\tau)w_{t+1} + (1-\lambda)b_{t+1}\right]\right) = \\          log\left(\beta\right) + log\left( \mathbb{E}_t \left[1+r_{t+1}\right]\right) + log\left( (1-s)(1-\tau)w_t \right)
\end{align*}
\begin{align*}
          log\left( \mathbb{E}_t\left[ s(1-\tau)(1+r_{t+1}) + \lambda\eta_{2,t+1}(1-\tau)w_{t+1} + (1-\lambda)b_{t+1} \right]\right) = \\          log\left(\beta\right) +  log\left( \mathbb{E}_t \left[1+r_{t+1}\right]\right) + log\left( (1-s)(1-\tau)w_t \right)
\end{align*}
We proceed by inserting the following expressions
\begin{itemize}
    \item $w_t =(1-\alpha)k_t^\alpha\zeta_t$ and $w_{t+1}=(1-\alpha)k_{t+1}^\alpha\zeta_{t+1}$
    \item $(1 +r_t) =\alpha k_t^{\alpha -1 }\zeta_t\eta_t$ and $(1+r_{t+1}) = \alpha k_{t+1}^{\alpha -1 }\zeta_{t+1}\eta_{t+1}$
    \item $b_t = \tau w_t \frac{1+\lambda}{1-\lambda} = ... =\tau (1-\alpha)k_t^\alpha\zeta_t \frac{1+\lambda}{1-\lambda}  $ and\\
    $b_{t+1} = \tau w_{t+1} \frac{1+\lambda}{1-\lambda} = ... \tau (1-\alpha)k_{t+1}^\alpha\zeta_{t+1} \frac{1+\lambda}{1-\lambda} $
\end{itemize}
\begin{align*}
          log\left( \mathbb{E}_t\left[ s(1-\tau)(\alpha k_{t+1}^{\alpha -1 }\zeta_{t+1}\eta_{t+1}) + \lambda\eta_{t+1}(1-\tau)(1-\alpha)k_{t+1}^\alpha\zeta_{t+1} + (1-\lambda)\tau (1-\alpha)k_{t+1}^\alpha\zeta_{t+1} \frac{1+\lambda}{1-\lambda} \right]\right) = \\          log\left(\beta\right) +  log\left( \mathbb{E}_t \left[\alpha k_{t+1}^{\alpha -1 }\zeta_{t+1}\eta_{t+1}\right]\right) + log\left( (1-s)(1-\tau)(1-\alpha)k_t^\alpha\zeta_t \right)
\end{align*}
All the shocks in expectation will equal 1, so that they cancel out:
\begin{align*}
          log\left( \mathbb{E}_t\left[ s(1-\tau)(\alpha k_{t+1}^{\alpha -1 }) + \lambda(1-\tau)(1-\alpha)k_{t+1}^\alpha+ (1-\lambda)\tau (1-\alpha)k_{t+1}^\alpha \frac{1+\lambda}{1-\lambda} \right]\right) = \\          log\left(\beta\right) +  log\left( \mathbb{E}_t \left[\alpha k_{t+1}^{\alpha -1 }\right]\right) + log\left( (1-s)(1-\tau)(1-\alpha)k_t^\alpha\zeta_t \right)
\end{align*}
Inserting $k_{t+1}=\frac{s(1-\tau)w_t}{1+\lambda}=\frac{s(1-\tau)(1-\alpha)k_t^\alpha\zeta_t}{1+\lambda}$ yields:
\begin{align*}
          log\left( \mathbb{E}_t\left[ s(1-\tau)\alpha \left( \frac{s(1-\tau)(1-\alpha)k_t^\alpha\zeta_t}{1+\lambda} \right)^{\alpha -1 } + \\
          \lambda(1-\tau)(1-\alpha)\left( \frac{s(1-\tau)(1-\alpha)k_t^\alpha\zeta_t}{1+\lambda}\right)^\alpha + \\
          (1-\lambda)\tau (1-\alpha)\left( \frac{s(1-\tau)(1-\alpha)k_t^\alpha\zeta_t}{1+\lambda} \right)^\alpha \frac{1+\lambda}{1-\lambda} \right]\right) = \\
          log\left(\beta\right) +  log\left( \mathbb{E}_t \left[\alpha \left( \frac{s(1-\tau)(1-\alpha)k_t^\alpha\zeta_t}{1+\lambda}\right)^{\alpha -1 }\right]\right) + \\
          log\left( (1-s)(1-\tau)(1-\alpha)k_t^\alpha\zeta_t \right)
\end{align*}

As can be seen from the matlab code exercise\_1\_3\_ABC.mat, our code works and as we increase the number of iterations the results for psi zero and psi one come closer and closer to the theoretical values (for both an aggregate recession or boom).

These are the final results for our PSI-array, in which the first line represents the constants and the later one the coefficient on the log of the lagged capital value.

Initial theoretical value of PSI:
\begin{pmatrix}
   -2.1982 &  -1.9367 \\
    0.3000  &  0.3000
\end{pmatrix}

Final regressed value of PSI (after $\>$100 iterations and T = 50,000):
\begin{pmatrix}
    -2.1244  & -1.9413 \\
    0.3139  &  0.3105 
\end{pmatrix}

    
\subsection{Section 1.4}
Different $\tau$ and EV: Interpret the findings. 
We adhered to the Krussel-Smith set-up with infinitely lived households.

\section{Summary of McKay and Reis (2016)}

McKay ad Reis (2016) presents a comprehensive analysis of the impact of automatics stabilizers on business cycle foundation. It has  been argued that automatic stabilizers - inherently counter-cyclical policies whose (main) objective is often not directly related to business cycle such as progressive taxation and transfers - represent a key component in mitigating macroeconomic fluctuations. In this paper, the authors constructed a Neo-Keynesian model with incomplete insurance markets - for a share of the population - capable of replicating the main feature of US business cycle while including all the automatic stabilizer present in the data and modelling all the four stabilization mechanism (disposable income, marginal incentives, redistribution, and social insurance). The latter are ensured by nominal rigidities, liquidity constrained consumers, incomplete insurance markets and precautionary saving, and strong intertemporal substitution. In particular, the automatic stabilizers analyzed in this paper are taxes (on personal income, property, sales, and corporate income), transfers (unemployment benefit and safety net payment), and government's budget deficit. Alongside idiosyncratic risk, the economy is affected by technological, monetary, and markup aggregate shocks. As the model cannot rely an a simple analysis of the aggregates - since distribution across agents matters - the authors used Reiter (2009) solution algorithm, preferred over Krusel and Smith´s. The calibration exercise with US data and a counterfactual study suggest that, in general, the effect of automatic stabilizer is weak. In particular, taxation appears to have a limited impact on volatility with an significant (and negative) impact on output and welfare. On the other hand, transfer to poor (liquidity constrained) and unemployed individuals were found to be rather impactful. Mechanism-wise, disposable-income is reported as being weak and the social insurance, despite being effective at stabilizing aggregate consumption, destabilizes aggregate output due to its interaction with changes in government deficit financing. However, the automatic stabilizer plays a useful role with sub-optimal monetary policy such at the zero-lower bound.

%\bibliography{project6.bib}
\pagebreak

\end{document}
