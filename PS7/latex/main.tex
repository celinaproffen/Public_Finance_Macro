\documentclass[12pt,a4paper]{article}

\usepackage[a4paper, top = 2cm, bottom = 2cm, left = 1.5cm, right = 1.5cm]{geometry}
\usepackage[dvipsnames]{xcolor} % Colors

\usepackage{standalone}

\usepackage{setspace}
\usepackage{graphicx}
\usepackage{amsfonts}
\usepackage{amsmath}
\usepackage{tikz}
\usepackage{pdfpages}
\usepackage{epigraph}
\usepackage{csquotes}
\usepackage{natbib}
\usepackage{accents}
\usepackage{pdfpages}

% Bibliography
\usepackage{xcolor}
\usepackage{hyperref}
\hypersetup{
colorlinks=true,
citecolor=MidnightBlue,
linkcolor=MidnightBlue,
pdfpagemode=FullScreen}

\usepackage{listings}
\lstset{frame=tb,
  language=Matlab,
  aboveskip=3mm,
  belowskip=3mm,
  showstringspaces=false,
  columns=flexible,
  basicstyle={\small\ttfamily},
  numbers=none,
  numberstyle=\tiny\color{gray},
  keywordstyle=\color{Red},
  commentstyle=\color{MidnightBlue},
  stringstyle=\color{Red},
  breaklines=true,
  breakatwhitespace=true,
  tabsize=3
}

\usepackage{natbib}
\usepackage[noabbrev]{cleveref}
\setcitestyle{authoryear,open={(},close={)}}
\bibliographystyle{plainnat}

\usepackage{subfiles}

\usepackage{url}
\urlstyle{same} % omit this command if you want monospaced-font look
\newcommand\purl[1]{\protect\url{#1}} % "protected url"

\setlength\parindent{0pt}
\spacing{1.2}

\begin{document}

\begin{center}
       \vspace*{4cm}
       \huge\textbf{Project 7} \\
       \vspace{0.4cm}
       \large \textbf{Public Finance in Macroeconomics} \\
       \vspace{0.5cm}
        \large Handed in by the \textcolor{orange}{\textbf{Heterogeneous Geeks}} \\
        \vspace{0.3cm}
        a.k.a. Vivien Voigt, Thong Nguyen, \includegraphics[scale=0.06]{geek.png}\\Davide Difino \& Celina Proffen \\
       \vspace{1.5cm}
       \vfill



        Project in the context of Prof. Ludwig's course: \\
        \textbf{Public Finance in Macroeconomics: Heterogenous Agent Models}\\
        at the Graduate School of Economics, Finance, and Management
       \vspace{0.8cm}
   \end{center}

\newpage

\section{Simple Variant of Krusell-Smith Algorithm}

\subsection{Characterization of Equilibrium Dynamics}

The proof of Proposition 3 uses a Guess and Verify approach.

\begin{enumerate}

  \item Guess that households will save a constant share $s$ of (disposable) wage,

    \begin{equation}
      a_{2,t+1} = s(1-\tau)w_t
    \label{guess}
    \end{equation}

  Since the firm optimization problem

    \[
      \max \Pi = \zeta_t F(K_t, \Upsilon_t L) - (\bar{\delta} + r_t)\varrho_t^{-1} K_t - w_t L,
    \]

  where $\zeta_t$ and $\varrho_t$ are aggregate shocks, imply the f.o.c.

    \[
      w_{t} = (1-\alpha)\Upsilon_t k_t^\alpha \zeta_t
    \]

  where

    \begin{equation}
        k_t = \frac{K_t}{\Upsilon_t L},
        \label{K_per_L}
    \end{equation}

  then \ref{guess} becomes

    \begin{equation}
        a_{2,t+1} = s(1-\tau)(1-\alpha)\Upsilon_t k_t^\alpha \zeta_t
    \label{a_guess}
    \end{equation}

  Market clearing conditions $K_t = a_{2,t}$ and $L = 1 + \lambda$, growth rate of labor productivity $\Upsilon_{t+1} = (1+g)\Upsilon_{t}$  and \ref{K_per_L} imply that

    \begin{equation}
        k_{t+1} = \frac{s(1-\tau)(1-\alpha) k_t^\alpha \zeta_t}{(1 + g)(1 + \lambda)}
        \label{K_guess}
    \end{equation}

  which is the equilibrium dynamics implied by our guess $s$.

  \item We know that, given $s$, \ref{K_guess} solves the Firm´s problem and the market clering condition. Then to verify that our guess indeed an equilibrium solution we need to check for which value of $s$ the household problem is solved.

  The agent budget constraints, substituting for $w_t$, $w_{t + 1}$ and $\alpha_{2, t+1}$ using \ref{a_guess} implies consumption levels

    \begin{align*}
      c_{1,t}& = (1 - s) w_t =(1 - s)(1 - \tau) \Upsilon_t k_t^\alpha \zeta_t  \\
      c_{2,t+1} & = s(1 - \tau) (1 - \alpha)\zeta_{t} \Upsilon_t k_t^\alpha \alpha \zeta_{t+1} \varrho_{t+1} k_{t+1}^{\alpha - 1}   \\
      & + (1 - \alpha) \Upsilon_{t+1}\zeta_{t+1} k_{t+1}^{\alpha} (\lambda \eta_{2, t+1} + \tau (1 + \lambda(1 - \eta_{2, t+1})))
    \end{align*}

  in the two periods lived by the agent, where the interest rate is provided by the second f.o.c of the firm,

  \[
    1 + r_t = \alpha k_t^{\alpha - 1} \zeta_{t} \varrho_{t}
  \]

  Substituting \ref{K_guess} yields

    \begin{align*}
     c_{2,t+1} & = (\alpha \varrho_{t+1} (1 + \lambda) + (1 + \alpha)(\lambda \eta_{2, t+1}  \\
     & + \tau(1 + \lambda(1 - \eta_{2, t+1})))) \Upsilon_{t+1}\zeta_{t+1} k_{t+1}^{\alpha}.
    \end{align*}

  Substituting the consumption levels $c_{1,t}$ and $c_{2,t+1}$ into the consumer Euler Equation

    \[
      1 = \beta \mathbb{E}_t \left[ \frac{c_{1,t} (1+r_{t+1})}{c_{2,t+1}} \right]
    \]

  implies

    \[
    1 =  \frac{\beta(1 - s)}{s}\Phi(\tau)
    \]

  where $\Phi(\tau)$ is a function of pension system contribution $\tau$ as defined in Proposition 3. Therefore, optimal saving rate in general quilibrium is also a function of $\tau$:

    \begin{equation}
      s(\tau) = \frac{\beta \Phi(\tau)}{1 + \beta \Phi(\tau)}.
      \label{opt_s}
    \end{equation}

  \item Due to convexity, the solution represented by \ref{K_guess} and \ref{opt_s} is unique.  Q.e.d.

\end{enumerate}


%\bibliography{project6.bib}

\pagebreak

\end{document}
